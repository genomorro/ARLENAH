\documentclass[10pt]{beamer}

\usetheme{Boadilla}

\usepackage[spanish]{babel}
\usepackage{lmodern}
\usepackage[utf8]{inputenc}
\usepackage{tikz} 
\usetikzlibrary{arrows,decorations.pathmorphing,backgrounds,fit,positioning,shapes.symbols,chains}

\beamersetuncovermixins{\opaqueness<1>{25}}{\opaqueness<2->{15}}

\begin{document}

\title{Archivo de lenguas de la ENAH}
\subtitle{Primeros apuntes sobre el Repositorio Digital}
\author{Karyn Galland \and \\
  Octavio González \and \\
  Edgar Domínguez}
\institute[ENAH]{Escuela Nacional de Antropología e Historia\\ENAH}
\date{22 de noviembre de 2013}

\begin{frame}[plain] 
  \titlepage
\end{frame}

\begin{frame}{ARLENAH}
  \framesubtitle{Visión como repositorio digital}
  
  \begin{block}{Misión}
    El Archivo  de Lenguas  de la ENAH  (ARLENAH) debe proporcionar  una plataforma  que organice,
    sistematice y  acumule información sobre lenguas  para ponerlas a disposición  de estudiantes,
    investigadores y público en general, como herramienta de custodia, difusión e investigación de
    los materiales socio-lingüísticos que producen los estudiantes, profesores e investigadores en
    las diferentes regiones lingüísticas en donde realizan sus actividades profesionales.    
  \end{block}


  \begin{block}{El software}
    DSpace es un software  de código abierto que proporciona organización  para capturar y describir
    material digital  usando módulos  de envío  con una  variedad de  opciones.  El  sistema permite
    distribuir informaciones  vía web de manera  que puedan ser recuperadas  mediante búsquedas, así
    como almacenar y preservar objetos digitales por periodos largos de tiempo.
  \end{block}  

\end{frame}

\begin{frame}{DSpace}
  \framesubtitle{Características}
  \begin{itemize}
  \item Desarrollado por  el Massachusetts Institute of Technology y  Hewlett-Packard Labs en el
    2002.
  \item Utilizado por: Universidad de Costa Rica, Cambridge University, University of Edinburgh,
    Université de Montréal.
  \item Código abierto bajo licencia Berkeley.
  \item Datos organizados en comunidades, colecciones, ítems.
  \item  Permite  capturar, describir,  buscar,  recuperar,  distribuir y  preservar  documentos
    digitales.
  \item Soporte de gran variedad de formatos.
  \item Set de metadatos.
  \item Robusta gestión de usuarios.
  \item Soporta  sistema operativo Linux, Solaris  y Windows. Requiere Servidor  de aplicaciones
    para Java como Apache Tomcat o Glassfish y base de datos PostgreSQL.
  \end{itemize}

\end{frame}


\begin{frame}{Dspace}
  \framesubtitle{Diagrama de funcionamiento}

  \tikzstyle{box}=[rectangle, draw, rounded corners, fill=yellow!20 , minimum width=3cm]
  \begin{figure}[H]
    \centering
    \begin{tikzpicture}
      \node [box] (comunity) {Comunidad};
      \node [box, below of=comunity, xshift=1cm] (collection) {Colección};
      \node [box, left of=collection, xshift=-3cm](e-person) {E-person};
      \node [box, below of=collection, xshift=-2cm] (item) {Item};
      \node [box, below of=item] (bundle) {Bundle};
      \node [box, below of=bundle] (bitstream) {Bitstream};

      \path[->,thick] 
      (comunity) edge (collection)
      (e-person) edge (item)
      (collection) edge (item)
      (item) edge (bundle)
      (bundle) edge (bitstream);
    \end{tikzpicture}    
  \end{figure}
\end{frame}


\begin{frame}{Descripción de los recursos}
  
  \begin{description}
  \item [Bitstream:] Representa un archivo específico subido al sistema DSpace.
  \item [Bundle:] Representa un  grupo de archivos que componen un  paquete.
  \item [Item:] Representa un Bundle acompañado de metadatos. Estos metadatos hacen posible la
    organización y búsqueda de recursos.
  \item [Colección:] Representa grupos de Items relacionados.
  \item[Comunidad:]  Es un  grupo de  colecciones que  comparte un  tema en  común.  Típicamente
    corresponde a una investigación, laboratorio o grupo de trabajo.
  \item [E-person:] Los usuarios con diferentes roles en DSpace.
  \end{description}

\end{frame}

\begin{frame}{Particularidades en ARLENAH}
  
  \begin{description}
  \item [Comunidad:] Estará formada por un grupo de investigadores que trabajan en un proyecto
    específico. La comunidad lleva  el nombre del proyecto o del  trabajo del investigador. La
    comunidad puede tener  subcomunidades que funcionan igual que las  comunidades pero que se
    utilizan cuando se requiere definir una extensión o especialización de la comunidad mayor.
  \item [Colección:] En el  caso del ARLENAH las colecciones llevan el nombre  de la lengua. Por
    lo tanto si el  proyecto abarca más de una lengua o diversas  variedades de lenguas existirá
    una colección  por cada una de  ellas. Una colección  no puede contener otras  colecciones y
    debe  estar contenida  siempre dentro  de  una comunidad.   Si la  comunidad desaparece  del
    repositorio también lo hará la colección.
  \item  [E-person:]   Existirán  tres   grupos  de   usuarios  con   diferentes  privilegios:
    Administrador de comunidad,  Remitentes, Visores.  Un usuario puede pertenecer  a un grupo
    diferente por cada comunidad en la que esté asociado.
  \end{description}

\end{frame}

\begin{frame}{ARLENAH}
  \framesubtitle{Organización de datos}

  \tikzstyle{box}=[rectangle, draw, rounded corners, fill=yellow!20, minimum width=3cm]
  \begin{figure}[H]
    \centering
    \begin{tikzpicture}
      \node [box] (comunity) {Comunidad: Proyecto Tzotzil};
      \node [box, below of=comunity] (collection) {Colección: Tzotzil};
      \node [box, below of=collection] (item-1) {Ítem: Flora};
      \node [box, right of=item-1, xshift=-5cm] (item-2) {Ítem: Parentesco};
      \node [box, right of=item-2, xshift=7cm] (item-3) {Ítem: Verbos};
      \node [box, below of=item-1, text width=3.3cm, yshift=-.2cm] (bundle-1) {TZO001R674I340.wav\\TZO001R674I341.wav};
      \node [box, below of=item-2] (bundle-2) {TZO001R674l427.wav};

      \path[->,thick] 
      (comunity) edge (collection)
      (collection) edge (item-1)
      (collection) edge (item-2)
      (collection) edge (item-3)
      (item-1) edge (bundle-1)
      (item-2) edge (bundle-2);
    \end{tikzpicture}    
  \end{figure}
\end{frame}

\begin{frame}{Actividades actuales}
  
  \begin{itemize}
  \item Seleccionar lista de metadatos obligatorios y optativos.
  \item Implementar filtros de verificación de información sobre los ítems y bitstreams.
  \item Implementar licencias y permisos sobre los ítems.
  \item Personalización gráfica del repositorio.
  \end{itemize}

\end{frame}

\end{document}
